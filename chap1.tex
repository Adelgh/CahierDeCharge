\newpage
\thispagestyle{empty}


\section{Introduction}

Le projet consiste à mettre en place un système de profiling. Ce système va être relié à un site web pour gérer les recommandations, c'est à dire qu'il va collecter les informations des visiteurs, les classifier en des groupes selon la ressemblance de leurs activités et leurs recommander des produits selon leurs préférences.

Définition de l'organisme.

l'objectif

\section{Travail demandé}

\subsection{Acteurs}

Les acteurs principaux de notre systèmes sont:


\begin{itemize}
    \item Le site web : cet acteur va être relié à notre système pour procurer à ses utilisateurs des recommandations selon leurs activités.
    
    \item Les internautes: Leurs activités va être tracées pour se bénéficier des recommandations qui simplifient leurs navigations.
\end{itemize}

\subsection{Besoins fonctionnels }
notre système doit fournir les besoins suivants:

\begin{itemize}
    \item Taguer les pages : il faut associer des tags aux pages selon lesquelles on va tracer l'activité des visiteurs.
    
    \item Tracer l'activité des internautes sur le site: On doit stocker ce que font les internautes pour déterminer les classes.
    
    \item Classifier : Ce système doit regrouper les internautes selon la ressemblance des visites.
    
    \item Recommander: Ce système doit recommander aux utilisateurs ce qu'ils peuvent faire sur le site pour leurs faciliter la navigation.
\end{itemize}

\subsection{Besoins non fonctionnels }

\begin{itemize}
    \item Les contraintes temporelles : Le développement doit respecter le cycle de vie du projet (les sprints) et doit respecter les délais.
    
    \item Contrainte de sécurité : 
    
    \item Contrainte de performance : Le temps de réponse doit être minimal.
\end{itemize}

\section{Environnements de travail }
\subsection{Environnement matériel }

\begin{itemize}
    \item {\textbf{PC :}} Dell
    \item {\textbf{Processeur :}} Intel(R) Core(TM) i7-4790 CPU @ 3.60GHz
    \item {\textbf{RAM :}} 20.0 Go
    \item {\textbf{Système d'exploitation :}} Système d'exploitation 64 bits, processeur x64

\end{itemize}
  \subsection{Environnement logiciel }  
  Pour la mise en place de notre système nous aurons recours principalement aux 3 langages de programmation:
  \begin{itemize}
      \item \textbf{Javascript} : pour taguer les pages. 
      \item \textbf{Java}: il sera utilisé pour le traçage des activités des visiteurs.
      \item \textbf{R}: pour l'implémentation des algorithmes de classification.
  \end{itemize}
  
  Pour cela nous utiliseront le framework : \textbf{Spring}.
  
  Comme IDE, nous travaillerons sur:
  \begin{itemize}
      \item Netbeans
      \item RStudio
  \end{itemize}

\section{Chronogramme prévisionnel du projet}
Le projet de déroulera pendant une durée de six mois s'étend sur la pèriode entre le 01 Février 2018 et le 31 Juillet 2018. La figure \ref{ch} illustre un planning prévisionnel, représentant les étapes principales permettant d'aboutir à une solution fonctionnelle répondant aux critères définis par le présent cahier des charges.
\begin{figure}[h]
\centering
\includegraphics[width=16cm,height= 4.5cm]{images/cover/diagrammedegantt.png}
\caption{Chronogramme du projet}
\label{ch}
\end{figure}